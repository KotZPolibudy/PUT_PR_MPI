\documentclass[11pt]{article}
\usepackage{lmodern,setspace,amsmath,amssymb,amsfonts,amsthm,graphicx,multicol,grffile,float}
\usepackage{authblk,url,csvsimple,cuted,dblfloatfix,parskip}
\usepackage[a4paper, top=0.9in, bottom=1.05in, left=1.01in, right=1.01in]{geometry}
\usepackage[polish]{babel}
\usepackage[utf8]{inputenc}
\usepackage[unicode]{hyperref}
\usepackage{listings}
\usepackage{booktabs}
\usepackage{fancyvrb}
\usepackage[T1]{fontenc}
\title{Przetwarzanie Rozproszone\\ Projekt -- Zabawa w zabijanie}
\author{Mikołaj Nowak 151813, Wojciech Kot 151879}
\date{Data oddania: \today}

\def\code#1{\texttt{#1}}

\begin{document}

\maketitle

\newpage

\section{Opis problemu}

W pewnej uczelni studenci zaczęli się bawić w zabójców i ofiary. W pierwszym cyklu student losuje rolę, następnie dobiera się w parę z ofiarą. Zabójca ubiega się o jeden z nierozróżnialnych P pistoletów na farbę. Następnie zabójca próbuje "zabić" ofiarę, a ofiara próbuje go uniknąć. Jednoznacznie należy rozstrzygnąć, jak skończył się dany cykl. Po X cyklach należy wyłonić zwycięzcę, po czym zabawa zaczyna się od nowa.

\section{Wiadomości}
Procesy mogą wysyłać do siebie wiadomości zawierające następujące informacje:
\begin{enumerate}
    \item Timestamp
    \item ID procesu nazawcy
    \item Wiadomość (INTEGER)
    \item Etykieta
    \begin{itemize}
        \item REQ: Request wejścia do sekcji krytycznej
        \item ACC: Accept (zezwolenie) na wejście do sekcji krytycznej
        \item FIN: Zakończenie jakiegoś etapu
        \item NONE: Pusta etykieta do zwykłej komunikacji np. wyłaniania zwycięzcy
    \end{itemize}
\end{enumerate}

\section{Struktury i zmienne}
Parametry:
\begin{itemize}
    \item N - liczba procesów
    \item P - liczba pistoletów dostępnych dla zabójców
    \item C - liczba cykli
    \item A - liczba kul (amunicji), które posiada każdy zabójca. 
\end{itemize}
Zmienne współdzielone:
\begin{itemize}
    \item Waiting - ID procesu czekającego na parowanie, początkowo -1 (zmienna używana
\end{itemize}
Zmienne prywatne dla każdego procesu:
\begin{itemize}
    \item WaitQueue - Kolejka procesów oczekujących na ACC, początkowo pusta
    \item AckNum - liczba otrzymanych potwierdzeń ACK, początkowo 0
    \item Partner - ID partnera
    \item Wynik - różnica liczby wygranych i przegranych rund
\end{itemize}

\section{Stany}
Początkowym stanem procesu jest REST
\begin{itemize}
    \item REST - nie ubiega się o dostęp.
    \item WAIT - czeka na dostęp do sekcji krytycznej
    \item INSECTION - w sekcji krytycznej
\end{itemize} 

\section{Szkic algorytmu}
Proces początkowo jest w stanie REST, a gdy ubiega się o dostęp do sekcji krytycznej w celu umieszczenia go w kolejce do przydzielenia pary zmienia go na WAIT. Wysyła REQ do wszytskich innych.\\
Reakcja na wiadomości:
\begin{itemize}
    \item REQ: wstaw żądanie do kolejki
    \item ACC: Zwiększ AckNum o 1
\end{itemize}
Następnie przydziel sobie partnera (INSECTION) oraz rolę sobie i partnerowi.\\
Wróć do stanu REST.\\
Wyślij ACC do reszty.\\
Rozpocznij rundę.\\
Zmień stan na WAIT, kiedy ubiega się o pisolet.\\
Wejdź do sekcji krytycznej (INSECTION) biorąc pistolet i próbując zabić przeciwnika.\\
Powtórz turę określoną liczbę razy.\\
Przekaż wiadomość FIN.

\section{Dobieranie procesów w pary}
Na samym początku zabawy procesy muszą się dobrać w pary (zabójca + ofiara).
\subsection{Wersja nieoptymalna}
Do dobierania procesów w pary można użyć algorytmu Lamporta z pojedyńczą sekcją krytyczną. W sekcji krytycznej znajduje się zmienna:
\begin{Verbatim}
int waiting := -1;
\end{Verbatim}
Oznaczająca ID procesu czekającego na partnera. Można założyć, że żaden proces nie może mieć ujemnego ID, dlatego na samym początku zmienna waiting przyjmuje wartość -1, bo nikt aktualnie nie czeka na parę.\\
Pierwszy proces wchodzi do sekcji krytycznej, ustawia wartość zmiennej waiting na swoje ID i następnie wychodzi z niej. Czeka on wtedy na odpowiedź od następnego procesu, który wkrótce stanie się jego parą.\\
Jeśli kolejny proces w sekcji krytycznej widzi, że zmienna waiting posiada zapamiętane jakieś ID, to zapamiętuje wartość tego ID, ustawia waiting na -1 i zwalnia jak najszybciej sekcję krytyczną.\\
Po opuszczeniu losuje sobie rolę (zabójca lub ofiara) i wysyła wiadomość do swojej pary, w której przekazuje swój identyfikator oraz rolę (partner będzie mieć przeciwną).

\begin{Verbatim}[numbers=left,xleftmargin=5mm]
global int waiting := -1;
local int time := 0;
local int ID;

function PAROWANIE():
    wyślij (REQ, time) do wszystkich poza mną
    if (odpowiedzi mają większe znaczniki czasowe OR żądanie na początku kolejki)
        wejdź do sekcji krytycznej
        if(waiting == -1):
            waiting = ID;
            wyjdź z sekcji krytycznej
            czekaj na wiadomość
        else:
            local int para := waiting;
            waiting := -1;
            wyjdź z sekcji krytycznej
            wylosuj rolę
            wyślij do para wiadomość o roli i swoim ID
\end{Verbatim}
\newpage

\subsection{Wersja uwzględniająca znajomość liczby procesów N}

Każdy proces wysyła REQUEST. Celem jest stworzenie N kolejek o długości N, gdzie N to oczywiście liczba procesów. \\
Kiedy kolejka osiągnie długość N, to znaczy, że dany proces zna dokładne uporządkowanie procesów, czyli kojeność, w której procesy wchodziłyby do sekcji krytycznej. Proces odnajduje swoją pozycję. Jeśli jest nieparzysta, to wysyła do procesu o 1 wcześniej w kolejce, że będzie jego parą i losuje role dla obu z pary. Proces parzysty czeka na wiadomość od nieparzystego. \\
Ponieważ procesy nie chcą wchodzić do sekcji krytycznej, to po znalezieniu się w kolejce usuwana jest zawartość RequestQueue.

\begin{Verbatim}[numbers=left,xleftmargin=5mm]
function PAROWANIE2():
    wyślij (REQ, time) do wszystkich poza mną
    odbieraj żądania innych dopóki długość WaitQueue jest mniejsza od N
    znajdź siebie w WaitQueue
    if(moja pozycja nieparzysta)
        Wylosuj sobie rolę
        Przekaż przeciwną partnerowi (z pocyji -1)
    else
        czekaj na rolę i na parę
\end{Verbatim}

\newpage
\section{Proces zabójcy}

N - ilość procesów \\
P - ilość pistoletów (zasobu) \\
A - ilość amunicji każdego zabójcy (prób które mogą podjąć) \\
state - zmienna przechowująca stan {RELEASED, REQUESTED, HELD, FINISHED} oznaczająca to czy obecny zabójca posiada broń, czy nie, albo czy już zakończył grę ze swojej strony. \\
C - ilość rund w cyklu \\

Kiedy proces zabójcy jest gotów do podjęcia próby zabójstwa zaczyna ubiegać się o dostęp do pistoletu (sekcji krytycznej) algorytmem Ricart'a i Agrawal'a.

Proces zabójcy:
\begin{Verbatim}[numbers=left,xleftmargin=5mm]
Proces sprawdza własną ilość amunicji (początkowo A)
    Jeśli jest większa od 0 - przechodzi dalej, 
    w przeciwnym razie zmienia stan na FINISHED i czeka na 
    koniec cyklu (odsyłając odpowiedzi na żądania dostępu do broni).

Następnie, jeśli ma jeszcze amunicje, ubiega się o dostęp do broni:
    Proces zmienia swój stan na REQUESTED
    Proces aktualizuje swój zegar i wysyła żądanie do wszystkich 
    pozostałych procesów, zawierając w nim swój znacznik czasu
    Proces czeka na odpowiedź od pozostałych procesów.

    Po uzyskaniu N-P odpowiedzi (od każdej ofiary oraz każdego zabójcy który
    nie ubiega się obecnie o broń) proces może przejść dalej - zabrać broń
    i podjąć się próby zabójstwa.

Po uzyskaniu odpowiedzi od ofiary proces zabójcy przechodzi dalej:
    W przypadku pomyślnego zabójstwa proces zabójcy zmienia stan na FINISHED
    i wysyła odpowiedzi na zakolejkowane żądania dostępu do broni.

    W przypadku niepowodzenia, proces zmienia stan na RELEASED, wysyła
    odpowiedzi na zakolejkowane żądania dostępu do broni i powtarza swoje
    zadanie (od kroku sprawdzenia dostępnej amunicji)

\end{Verbatim}

Próba zabójstwa (przykładowa)
\begin{Verbatim}[numbers=left,xleftmargin=5mm]
Proces zmniejsza swój licznik amunicji o 1
Proces wysyła do swojej ofiary informacje o podjętej próbie z
wylosowaną wartością z zakresu 1-20
Ofiara losuje własną liczbę od 1-20 określającą trudność tej próby
    Jeśli liczba zabójcy jest większa niż liczba ofiary -> ofiara przesyła
    informacje o powodzeniu i zmienia własny stan z ALIVE na DEAD

    W przeciwnym wypadku ofiara odsyła wiadomość o niepowodzeniu.

Proces zabójcy odbiera wiadomość i przechodzi dalej w swojej implementacji
\end{Verbatim}

\newpage
%%\subsection{Wykrywanie stanu końcowego}
\section{Wykrywanie stanu końcowego}

Kiedy proces zabójcy zmienia swój stan na Finished wysyła do kolejnego (w pierścieniu) wiadomość FIN zawierającą jego numer procesu oraz dwie zmienne ilość udanych zabójstw oraz ilość nieudanych zabójstw (początkowo 0,0). Każdy proces który jest już w stanie FINISHED przekazuje wiadomości FIN dalej z tym samym numerem procesu który ją wysłał, zwiększając odpowiedni licznik (udanych lub nieudanych) zabójstw o 1.
Gdy proces otrzyma wiadomość FIN z własnym numerem jako nadawcy również zwiększa odpowiedni licznik ale nie przekazuje jej dalej, zamiast tego zwiększa licznik rund oraz zapisuje w tablicy o długości C wynik odpowiedniej rundy. 
W momencie gdy licznik rund osiągnie C należy podsumować cykl, a więc każdy proces wykonuje print() wyświetlając ogólny wynik cyklu 

Wygrana Zabójców gdy więcej rund zostało wygranych przez zabójców
Wygrana ofiar gdy większość rund została przegrana przez zabójców
Remis, gdy obie wielkości będą równe (np. gdy wszystkie rundy zakończyły się remisami)

Po podsumowaniu gra zaczyna się ponownie, procesy wracają do dobierania się w pary i rozpoczynają nowy cykl C rund.


\section{Program główny}

\begin{Verbatim}[numbers=left,xleftmargin=5mm]

Powtarzaj ciągle (While true:)
    Znajdź swoją parę i wylosujcie role w tym cyklu
    Rozpocznij cykl gry (powtarzaj C razy):
        Jeśli jesteś samotny, zachowuj się jak ofiara.
        Jeśli jesteś ofiarą czekaj na sygnały i odpowiadaj na nie odpowiednio
        Jeśli jesteś zabójcą:
            Podejmij próbę zabicia swojej ofiary.
            Zbierz wynik tej rundy i zapisz
    Po skończeniu cyklu, jeśli byłeś zabójcą podsumuj cykl
    (podlicz wygrane i przegrane i wyświetl wynik)
    Ponów grę w nieskończoność.
\end{Verbatim}

Należy zauważyć że przy nieparzystej ilości graczy, jeden z procesów zostanie zablokowany na etapie parowania. Natomiast nie jest to problem, gdyż wystarczy żeby zachowywał się jak ofiara dla komunikatów końca i czekał na kolejną fazę parowania -> ma wtedy zapewnione miejsce w pierwszej parze nowego cyklu.



\end{document}
